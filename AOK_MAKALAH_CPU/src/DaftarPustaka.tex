Architecture of the central processing unit (CPU). https://computersciencewiki.org/
\linebreak
index.php/Architecture
\_of\_the\_central\_processing\_unit\_(CPU),(9 Oktober 2022).

ashvika99.Difference between Cache Memory and Register.
https://www.geeksforgeeks.org/
difference-between-cache-memory-and-register,[diakses pada 16 Februari 2024].

Azimah, A., Sucahyo, Y. G., dan Komputer, I. F. I. (2007). Penggunaan Data Warehouse dan Data Mining untuk Data Akademik. J. Sist. Inf. MTI UI, 3(2), 1-7.

Encyclopedia Brittanica.Cache Memory.https://www.britannica.com/technology/cache-memory,[diakses pada16 Februari 2024].

Encyclopedia Brittanica.Central Processing Unit.https://www.britannica.com/technology/central-processing-unit,[diakses pada 16 Februari 2024].

Encyclopedia Brittanica.Control Unit.https://www.britannica.com/technology/control-unit,[diakses pada 16 Februari 2024].

Eswaran,Suwarthy.Arithmetic Logic Unit Design.https://witscad.com/course/computer-architecture/chapter/arithmetic-logic-unit-design,(diakses pada 16 Februari 2024).

geeksforgeeks.Little and Big Endian Mystery.https://www.geeksforgeeks.org/little-and-big-endian-mystery,[diakses pada 16 Februari 2024].

Gilbert, Howard.CPU Instructions.(13 Februari).https://pclt.sites.yale.edu/cpu-instructions,(diakses pada 16 Februari 2024).

Kumar, Viskas.Understanding L1, L2, and L3 Caches: How to Improve CPU Performance.(21 September 2023).https://levelup.gitconnected.com/understanding-l1-l2-and-l3-caches-how-to-improve-cpu-performance-d9dcc3e2e1f5,(diakses pada 16 Februari 2024).

Memory - Byte Order - (Endian-ness) of a Word.(11 December 2023).
\linebreak
https://datacadamia.com/computer/memory/endian,(diakses pada 16 Februari 2024).

Patterson, David. A, dan Hennessy, John L.2012.Computer Organization and Design: The Hardware/Software Interface. Massachusetts:Morgan Kaufmann.

Prasetyo, B., Puspitasari, A., dan Nasution, R.2019. Implementasi Manajemen Bandwidth Dan Filtering Web Access Control Menggunakan Metode Address List. JIKA (Jurnal Informatika), 3(2), 156-165.

Setyoadi, Y., dan Latifah, K. (2015). Integrasi Software CAD-CAM dalam Sistem Operasi Mesin Bubut CNC. Jurnal Informatika Upgris, 1(2 Desember).

Tanenbaum. Andrew S, dan Austin, Todd.2012.Structured Computer Organization.Michigan:Pearson.

Types of Register in Computer Organization.https://www.javatpoint.com/types-of-register-in-computer-organization,[diakses pada 16 Februari 2024].

Wang, X., Fu, X., Liu, X., dan Gu, Z. (2009, April). Power-aware cpu utilization control for distributed real-time systems. In 2009 15th IEEE Real-Time and Embedded Technology and Applications Symposium (pp. 233-242).

Wilson, Kevin.2017.Computer Hardware: The Illustrated Guide to Understanding Computer Hardware.Widnes:Elluminet Press.

x86 Registers.
https://www.eecg.utoronto.ca/~amza/www.mindsec.com/
\linebreak
files/x86regs.html,[diakses pada 16 Februari 2024].
