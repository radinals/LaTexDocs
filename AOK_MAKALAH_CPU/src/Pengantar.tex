Puji syukur diucapkan ke hadirat Allah Swt. atas segala rahmat-Nya sehingga
makalah ini dapat tersusun sampai dengan selesai.

Makalah ini dibuat untuk memenuhi persyaratan tugas mata kuliah Arsitektur dan
Organisasi Komputer di Fakultas Teknik Informatika Universitas Suryakancana.

Semoga makalah ini dapat memberikan kontribusi positif
terhadap pemahaman Anda tentang CPU dalam konteks Arsitektur dan Organisasi Komputer.
Kami mengucapkan terima kasih kepada dosen dan pembimbing kami atas panduan dan dukungan
yang diberikan selama proses penulisan makalah ini

Pada makalah ini akan dibahas tentang \textit{Central Processing Unit} (\textit{CPU})
atau disebut juga dengan prosesor, komponen-komponen yang terdapat didalamnya,
dan bagaimana komponen tersebut berkerja sama dalam suatu organisasi arsitektur CPU.

Bagi kami sebagai penyusun merasa bahwa masih banyak kekurangan dalam
penyusunan makalah ini karena keterbatasan pengetahuan dan pengalaman Kami.
Untuk itu kami sangat mengharapkan kritik dan saran yang membangun dari
pembaca demi kesempurnaan makalah ini.

\vspace{1cm}

\begin{center}
  \begin{flushright}
    Cianjur, 16 Februari 2024

    Penyusun
  \end{flushright}
\end{center}
