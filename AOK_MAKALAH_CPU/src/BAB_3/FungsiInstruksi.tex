Secara historis, di awal mula komputer para peneliti menggunakan nilai-nilai biner
yang di inputkan ke dalam sistem komputer satu persatu agar dapat memprogramnya.

Dan dengan berkembangya jaman, serta berkembangya permintaan atas komputer yang
lebih kompleks, metode pemograman dengan angka biner sudah tidak lagi efisien
dan sangat beresiko untuk membuat kesalahan.

Maka diciptakan bahasa pemograman atau high-level, yang diartikan sebagai bahasa
yang secara sintaks berada mendekati ke bahasa manusia (high) dibandingkan ke
bahasa mesin atau biner (low).

Dan dijaman sekarang pun, meski sudah ada bahasa yang secara sintaks sudah sangat
menyerupai bahasa manusia, masih ada kasus-kasus dimana bahasa mesin atau setidaknya
bahasa assembly lebih efektif.

Untuk saat ini, bahasa low-levell (mendekati mesin) yang paling umum untuk digunakan
adalah bahasa assembly, dan hal ini beralasan, karena semakin tinggi suatu bahasa
atau mendekati bahasa manusia, ada hal yang harus dikorbankan, yaitu kecepatan.

Maka untuk dimana kecepatan itu sangat penting, seperti untuk Driver Hardware
(Microcode, atau software yang memastikan hardware dapat dijalankan sistem), ataupun
embeded computer alias mesin-mesin komputer yang kecil dan tidak memiliki ruang
atau pun kekuatan yang cukup untuk menjalankan program yang mungkin lebih besar dari
1 KiloByte.

Jadi, di masa kini pun mesik sudah ada bahasa pemograman yang jauh lebih mudah
digunakan dan lebih mudah untuk dipakai untuk membuat aplikasi, permintaan untuk
programmer yang paham pemograman tingkat mesin pun akan selalu ada.
