Set Instruksi, atau Set Instruksi CPU (ISA) adalah perintah-perintah yang berfungsi untuk
memperintah CPU agar melakukan sesuatu, entah itu untuk melakukan kalkulasi aritmatika,
kalkulasi boolean, atau untuk melakukan suatu proses input-output di suatu peripheral komputer.

Menurut David A. Patterson dan John L. Hennessy di dalam buku "Computer Organization and Design",
``ISA adalah kontrak antara perangkat keras dan perangkat lunak, menentukan instruksi apa yang
dapat dikelola oleh prosesor dan operasi apa yang dilakukan setiap instruksi.``

jadi dapat disimpulkan dari kutipan tadi. Tanpa adanya set instruksi,
sebuah komputer tidak akan bisa melakukan apapun, karena tanpa adanya instruksi
komponen sofware suatu komputer tidak akan bisa ada, dan tanpa software, tidak akan
ada yang mengendalikan hardware tersebut.
