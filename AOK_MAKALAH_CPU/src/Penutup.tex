\subsection{Kesimpulan}

Kesimpulan yang didapat adalah, dalam CPU atau Central Processing Unit terdapat
tiga unit terpenting yaitu,

\begin{itemize}
  \item Control unit
  \item Arithmetic/Logic Unit (ALU)
  \item Memory Unit
\end{itemize}

Dengan control unit bertugas untuk mengatur dua unit lainnya, Arithmetic/Logic Unit
atau ALU bertugas untuk melakukan perhitungan-perhitungan yang diperlukan dan proses
ini dibantu dengan Memory Unit yang terdiri dari cache dan register, fungsi utama dari
unit ini adalah untuk menyediakan tempat penyimpanan untuk data-data yang sedang di proses CPU.

Dan di tingkat mikro, terdapat istilah operasi mikro yang mengacu kepada operasi tingkat
bit yang dilakukan oleh gerbang-gerbang logika yang berada di control unit CPU. Operasi Mikro
berasal hasil representasi instruksi yang disampaikan kepada CPU ketika sebuah program berjalan
representasi ini berbentuk biner dan disebut dengan opcode atau Operation-Code.

\subsection{Saran}

Agar lebih memahami operasi yang berada di tingkat CPU, perlu juga dipelajari
pemograman di tingkat rendah yang berbentuk assembly. Dengan memahami proses
eksekusi program di tingkat assembly, pemahaman bagaimana sebuah instruksi diproses
menjadi opcode akan didapatkan.
