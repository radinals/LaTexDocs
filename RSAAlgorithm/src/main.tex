\documentclass[11pt, a4paper]{article}
\usepackage[T1]{fontenc}
\usepackage{graphicx}
\usepackage{mathptmx}
\usepackage{listings}
\usepackage{xcolor}
\usepackage{ragged2e}
\usepackage{amsmath}
\usepackage{amssymb}
\usepackage{amsfonts,amssymb}

\date{}

\begin{document}

\section* {Langkah-Langkah Pembuatan Key}

  \begin{enumerate}
    \item Menentukan Nilai P dan Q

      Langkah awal untuk membuat key memerlukan
      dua buah angka prima yang besar yaitu P dan Q, dimana P $\neq$ Q.

    \item Menentukan Nilai N

      Setelah Menentukan nilai P dan Q, dapatkan nilai N dari rumus

      N = P $\cdot$ Q

      Nilai N bisa dipublikasikan, N ini akan menjadi bagian dari private dan public key.

    \item Menentukan Nilai M

      Dan setelah mendapat nilai N, diperlukan juga nilai M. M didapatkan dari rumus
      berikut.

      M = $\phi$(N), atau M = (P - 1) $\cdot$ (Q - 1)

    \item Menentukan Nilai E

      Lalu Nilai E, nilai ini digunakan dalam operasi enkripsian. Nilai E adalah
      angka yang dipilih secara acak dan adalah relatif prima dengan nilai M.

      Nilai E ini bersama dengan nilai N membentuk Public Key.

    \item Menentukan Nilai D

      Terakhir Nilai D, nilai ini digunakan dalam operasi dekripsi. Nilai D adalah
      angka yang memenuhi kongruen berikut.

       E $\cdot$ D $\equiv$ 1 (mod M), atau D = $\frac{1 + (k + M)}{E}$

      Nilai D ini bersama nilai N membentuk Private Key

  \end{enumerate}
\newpage
\section*{Contoh Pembuatan Key}

Dengan P = 23 dan Q = 29. Maka didapatkan,

\begin{center}
N = 23 $\cdot$ 29 = 667

M = (23 - 1) $\cdot$ (29 - 1) = 616
\end{center}

Nilai E dapat dipilih secara acak selama nilai tersebut merupakan relatif prima
dari nilai M yaitu 616.

Program memberikan nilai 127. kita bisa memastikan ini relatif prima dengan
mengecek hasil dari gcd(127,616)

\begin{center}
616 = 4 $\cdot$ 127 + 8

127 = 15 $\cdot$ 8 + 7

8 = 1 $\cdot$ 7 + 1

7 = 7 $\cdot$ 1 + 0
\end{center}

Dan karena pembagi terakhir adalah 1, jadi gcd(127,616) adalah 1, artinya
127 dan 616 adalah relatif prima.

\vspace{0.5cm}

Lalu Nilai D dioutputkan program sebagai 519, untuk membuktikan ini benar
kita dapat mengecek kekongruenan berikut.

\begin{center}
127 $\cdot$ 519 $\equiv$ 1 (mod 616)
\end{center}


Hasil dari 127 dikali 519 yaitu 65913 harus bernilai 1 jika dibagi 616, seperti yang
dapat dilihat sebagai berikut.

\begin{center}
65913 mod 616 = 107, sisa 1
\end{center}

Dari hasil diatas, dapat disimpulkan bahwa nilai yang dihasilkan program, 519
adalah benar karena memenuhi kekongruenan.

\newpage

\section*{Proses Dekripsi/Enkripsi}

\begin{center}

Enkripsi: C(P) = $P^{E}$ mod N

Dekripsi: P(C) = $C^{D}$ mod N

\end{center}

\begin{itemize}
  \item C = cyphertext.
  \item P = plaintext.
\end{itemize}

\section*{Contoh Proses Dekripsi/Enkripsi}

Dengan menggunakan nilai yang telah dihasilkan sebelumnya, terdapat nilai berikut.

\begin{center}
N = 667

E = 127

D = 519
\end{center}

Misalkan terdapat Plaintext yaitu "Budi".

Plaintext ini jika diubah menjadi nilai ASCII akan menjadi, (66,117,100,105).

Program akan melakukan enkripsi/dekripsi pada tiap huruf, hingga proses enkripsi
akan berjalan sebagai berikut.

\begin{center}
B = $(66)^{127}$ mod 667 = 398

u = $(117)^{127}$ mod 667 = 639

d = $(100)^{127}$ mod 667 = 13

i = $(105)^{127}$ mod 667 = 98
\end{center}

Nilai yang dihasilkan proses enkripsi biasanya adalah angka yang sangat besar,
hingga diluar jangkauan nilai ASCII, untuk kasus ini hasi enkripsi menghasilkan nilai
(398, 639, 13, 98).

Lalu dalam proses dekripsi, dilakukan juga secara per huruf.

\begin{center}
398 = $(398)^{519}$ mod 667 = 66 (B)

639 = $(639)^{519}$ mod 667 = 117 (u)

13 = $(13)^{519}$ mod 667 = 13 (d)

98 = $(98)^{519}$ mod 667 = 105 (i)
\end{center}

Setelah proses dekripsi, bisa dilihat akan didapatkan kembali nilai awal,
ini menandakan kedua proses berhasil.

\end{document}
