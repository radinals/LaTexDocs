\documentclass[11pt, a4paper]{article}
\usepackage[T1]{fontenc}
\usepackage{graphicx}
\usepackage{mathptmx}
\usepackage{listings}
\usepackage{xcolor}
\usepackage{ragged2e}
\date{}
% \pagenumbering{gobble}

\graphicspath{{./img}}

\definecolor{mGreen}{rgb}{0,0.6,0}
\definecolor{mGray}{rgb}{0.5,0.5,0.5}
\definecolor{mPurple}{rgb}{0.58,0,0.82}
\definecolor{backgroundColour}{rgb}{0.95,0.95,0.92}
\vbadness=1000
\overfullrule=1000mm
\hfuzz=1000pt
\vfuzz=20pt

\lstset{
  language=C,                     % choose the language of the code
  numbers=left,                   % where to put the line-numbers
  stepnumber=1,                   % the step between two line-numbers.
  numbersep=10pt,                 % how far the line-numbers are from the code
  backgroundcolor=\color{white},  % choose the background color. You must add \usepackage{color}
  showspaces=false,               % show spaces adding particular underscores
  showstringspaces=false,         % underline spaces within strings
  showtabs=false,                 % show tabs within strings adding particular underscores
  tabsize=2,                      % sets default tabsize to 2 spaces
  captionpos=b,                   % sets the caption-position to bottom
  breaklines=true,                % sets automatic line breaking
  breakatwhitespace=true,         % sets if automatic breaks should only happen at whitespace
  % title=\lstname,               % show the filename of files included with \lstinputlisting;
}

\begin{document}

     \begin{center}
         \vspace*{1cm}

         \textbf{\Huge{SimpleRSA}}

         \vspace{0.2cm}

         Implementasi Sederhana Algoritma RSA menggunkanan C++

         \vspace{0.2cm}

     \end{center}

  \pagenumbering{arabic}

  Program SimpleRSA di backend menyimpan cyphertext/plaintext di dalam objek khusus,
  yaitu RSAText seperti sebagai berikut.

    \subsubsection*{Header (rsatext.h)}
      \lstinputlisting{source_file/rsatext.h}

    \newpage

    \subsubsection*{Source (rsatext.cpp)}
      \lstinputlisting{source_file/rsatext.cpp}

  nilai string, maupun itu plaintext atau cyphertext akan dimanipulasi oleh objek SimpleRSA,
  string tersebut akan di pisahkan (split) per karakter dan kemudian akan di proses oleh
  SimpleRSA untuk di dekripsi atau di enkripsi.

  Perlu dicatat tidak ada pengecekan internal dari kedua objek ini yang akan
  memastikan suatu plaintext atau cyphertext akan diproses lebih dari seharusnya.

  Jadi pemakai objek ini harus mengingat apakah string yang tersimpan
  itu telah di dekripsi atau telah terenkripsi.

   \vspace{0.2cm}

    \subsubsection*{Header (simplersa.h)}
      \lstinputlisting{source_file/simplersa.cpp}

    \newpage

    \subsubsection*{Source (simplersa.cpp)}
      \lstinputlisting{source_file/simplersa.cpp}

\end{document}
