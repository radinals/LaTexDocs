\documentclass[12pt,a4paper]{article}

\usepackage{geometry}
\usepackage{graphicx}
\usepackage{ragged2e}
\usepackage{multicol}
\usepackage{adjustbox}

\usepackage{listings}
\usepackage{xcolor}

\lstset{
  basicstyle=\ttfamily\normalfont\scriptsize,
  breakatwhitespace=true,         
  escapeinside={\%*}{*)},
  breakautoindent=true,
  breaklines=true,                 
  captionpos=b,                    
  keepspaces=false,                 
  showspaces=false,                
  showstringspaces=false,
  showtabs=false,                  
  frame=single,
  numbers=none,
  stepnumber=1,% the step between two line-numbers. If it's 1 each line will be numbered
  tabsize=2
}


% Bash
\lstdefinelanguage{Bash}{
  morekeywords={
    if,then,else,elif,fi,for,while,do,done,case,esac,export
  },
  sensitive=true,
  morecomment=[l]\#,
  morestring=[b]",
}

% C
\lstdefinelanguage{C}{
  morekeywords={
    auto,break,case,char,const,continue,default,do,double,else,
    enum,extern,for,if,int,long,register,return,switch,typedef,
    unsigned,void,volatile,while
  },
  sensitive=true,
  morecomment=[l]//,
  morecomment=[s]/* */ ,
  morestring=[b]",
}

% C++
\lstdefinelanguage{C++}{
  morekeywords={
    alignas,alignof,and,asm,auto,bitand,bitor,bool,break,case,
    class,compl,const,constexpr,continue,decltype,default,delete,
    do,double,else,enum,explicit,false,for,friend,goto,if,inline,
    int,long,mutable namespace,new noexcept,operator,private,
    protected,public,register,return,short,signed,sizeof,
    static,static_assert,static_cast,struct,switch,template,this,
    thread_local,throw,true,try,typedef,typeid,typename,union,
    unsigned,using,virtual,void,volatile,wchar_t,while
  },
  sensitive=true,
  morecomment=[l]//,
  morecomment=[s]/* */ ,
  morestring=[b]",
}

% Java
\lstdefinelanguage{Java}{
  morekeywords={
    abstract,assert,boolean,break,byte,case,catch,char,class,
    const,continue,default,do,double,else,enum,extends,final,
    finally,float, for,goto,if,implements,import,instanceof,int,
    interface,long,native,new, null,package,private,protected,
    public,return,short,static,strictfp, super,switch,
    synchronized,this,throw,throws,transient,true,try,void,
    volatile,while
  },
  sensitive=true,
  morecomment=[l]//,
  morecomment=[s]/* */ ,
  morestring=[b]",
}

% Go
\lstdefinelanguage{Go}{
  morekeywords={
    break,case,chan,const,continue,
    default,defer,else,fallthrough,
    for,function,goto,if,import,interface,
    map,package,range,return,select,struct,
    switch,type,var},
  sensitive=true,
  morecomment=[l]//,
  morecomment=[s]/* */ ,
  morestring=[b]",
}

% PHP
\lstdefinelanguage{PHP}{
  morekeywords={
    __halt_compiler,abstract,alias,arguments,break,case,class,
    clone,const,continue,declare,default,die,do,echo,else,elseif,
    empty,endswitch,eval,exit,extends,final,finally,for,foreach,
    function,global,goto,if,implements,include,include_once,
    instanceof,insteadof,interface,is,isset,list,namespace,
    print,private,protected,public,return,static,switch,throw,
    trait,try,unset,use,var,while,yield
  },
  sensitive=true,
  morecomment=[l]//,
  morecomment=[s]/* */ ,
  morestring=[b]",
}

% Javascript
\lstdefinelanguage{JavaScript}{
    morekeywords={abstract,arguments,await,boolean,break,byte,
    case,catch,class,const,continue,debugger,default,delete,do,
    double,else,enum,eval,export,extends,false,finally,for,function,
    global,if,implements,import,in,instanceof,int,let,match,namespace,
    NaN,private,protected,public,return,super,switch,throw,throws,true,
    try,typeof,var,void,yield
  },
  sensitive=true,
  morecomment=[l]//,
  morecomment=[s]/* */ ,
  morestring=[b]",
}



\geometry{margin=1cm}
\graphicspath { {./img/} }

\pagenumbering{gobble}
\date{}

\begin{document}

Radinal Shidiq Saragih

IF C 2023 5520123104

\vspace{1cm}

\textbf{Sinyal Analog}
\begin{itemize}
    \item Frekuensi sinyal analog: $f = 1 \, \text{kHz}$.
    \item Amplitudo maksimum sinyal: $A = 1.0 \, \text{V}$.
    \item Frekuensi sampling: $f_s = 8 \, \text{kHz}$.
    \item Bit-depth kuantisasi: $\text{Bit-depth} = 3 \, \text{bit}$ (8 tingkat kuantisasi).
\end{itemize}
    
\begin{table}[h!]
\centering
\begin{tabular}{|c|c|c|c|c|}
\hline
\textbf{Waktu ($t_n$) [s]} & \textbf{Amplitudo Sinyal ($x(t)$) [V]} & \textbf{Level Kuantisasi} & \textbf{Kuantisasi ($Q$) [V]} & \textbf{Biner} \\ \hline
0.0000                     & 0.000                                 & Level 5                  & 0.25                          & 101            \\ \hline
0.000125                   & 0.707                                 & Level 7                  & 0.75                          & 111            \\ \hline
0.000250                   & 1.000                                 & Level 8                  & 1.00                          & 111            \\ \hline
0.000375                   & 0.707                                 & Level 7                  & 0.75                          & 110            \\ \hline
0.000500                   & 0.000                                 & Level 5                  & 0.25                          & 101            \\ \hline
0.000625                   & -0.707                                & Level 3                  & -0.75                         & 011            \\ \hline
0.000750                   & -1.000                                & Level 1                  & -1.00                         & 000            \\ \hline
0.000875                   & -0.707                                & Level 3                  & -0.75                         & 011            \\ \hline
\end{tabular}
\end{table}

\textbf{Kompresi}

\begin{itemize}
  \item Data(Biner) : [101, 111, 111, 110, 101, 011, 000, 011]
  \item Data (Decimal) : [ 5, 7, 7, 6, 5, 3, 0, 3 ]
  \item Terdapat 8 nilai.
  \item Bit depth 3 bit.
\end{itemize}

Total Ukuran data asli:

\begin{center}
  Ukuran Awal = 8 x 3 = 24 bit
\end{center}

Dikompresi dengan Metode Kuantisasi

dengan 4 tingkat kuantisasi, yaitu

\begin{center}

  0 - 1 = 1 

  2 - 3 = 4 

  4 - 5 = 6 

  6 - 7 = 8

\end{center}

Data (Setelah Kuantisasi) = [ 6, 8, 8, 8, 6, 4, 1, 3 ]

\vspace{1cm}

Karena ada 4 tingkatan kuantisasi, maka hanya memerlukan 2 bit untuk
menyimpan data. Maka ukuran setelah kuantisasi adalah

\begin{center}
  Ukuran Setelah Kuantisasi = 8 x 2 = 16 bit
\end{center}

Hingga rasio kompresinya adalah

\begin{center}
  Rasio Kompresi = ((24 - 16) / 24) x 100\% = 50\%
\end{center}

Jadi setelah dikompresi data menjadi 50\% lebih kecil dari sebelumnya.

\end{document}
