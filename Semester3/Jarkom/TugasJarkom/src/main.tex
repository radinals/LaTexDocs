\documentclass[12pt, a4paper]{article}

\usepackage[T1]{fontenc}
\usepackage{courier}
\usepackage{graphicx}
\usepackage{caption}
\usepackage{mathptmx}
\usepackage{ragged2e}
\usepackage{xcolor}
\usepackage{listings}
\usepackage{enumitem}

\graphicspath{{img}}
\pagenumbering{gobble}
\date{}

\captionsetup[figure]{labelformat=empty}

\newcommand{\addSection}[2] {
  \begin{center}
    \large{\textbf{#1}}
    \section*{#2}
  \end{center}
  \addcontentsline{toc}{section}{#1 #2}{}
  \setcounter{section}{2}
  \setcounter{subsection}{0}
  \vspace{1cm}
}

\title{

  \LARGE{\textbf{LAPORAN}}

  {\vspace{1cm}}

  \large{\textbf{Eksplorasi Pembuatan FOOBAR}}

  {\large{Diajukan untuk Memenuhi Tugas Mata Kuliah Jaringan Komputer}}

  {\vspace{1cm}}

  \normalsize{Radinal Shidiq Saragih}

  \normalsize{Mohammad Hiqmal Ariffansyah}

  \normalsize{Ridho Faatihul Ihsan}

  {\vspace{1cm}}

  {\includegraphics[scale=1.8]{LogoFakultas.jpeg}}

  {\vspace{2cm}}

  {\large{PROGRAM STUDI TEKNIK INFORMATIKA}}

  {\large{FAKULTAS TEKNIK}}

  {\large{UNIVERSITAS SURYAKANCANA}}

  {\large{CIANJUR}}

  {\small{2024}}
}

\begin{document}
  \begin{titlepage}
    \maketitle
  \end{titlepage}

  \pagenumbering{roman}

  %-KATA PENGANTAR-----------------------------------------------------------------
  \begin{center}
    \section*{KATA PENGANTAR}
  \end{center}
  \setcounter{section}{1}
  \setcounter{subsection}{0}
  \addcontentsline{toc}{section}{KATA PENGANTAR}{}
  \vspace{1cm}
  \begin{flushright}
    Cianjur, September 2024

    \vspace{0.5cm}

    Penulis
  \end{flushright}
  %---------------------------------------------------------------------------

  \newpage

  %-TOC-----------------------------------------------------------------------
  \renewcommand\contentsname {\Large{\textbf{DAFTAR ISI}} }
  \begin{center}
    \tableofcontents
  \end{center}
  \addcontentsline{toc}{section}{DAFTAR ISI}{}
  \vspace{1cm}
  %---------------------------------------------------------------------------

  \newpage

  %---------------------------------------------------------------------------

  \pagenumbering{arabic}

  %---------------------------------------------------------------------------
  \addSection{BAB I}{PENDAHULUAN}

  \subsection{Latar Belakang}

  \subsection{Rumusan Masalah}

  \subsection{Tujuan}

  %---------------------------------------------------------------------------
  \newpage
  %---------------------------------------------------------------------------

  \addSection{BAB II}{ANALISIS DAN DESIGN}

  \subsection{Design Topologi}

  Topologi yang dibangun di simulasi terdiri dari empat buah jaringan yang
  terhubung pada satu router, dengan router tersebut nantinya akan
  terhubung dengan router lainnya dalam sebuah rangkaian ring.

  Alasan utama dari penggunaan lebih dari satu router di sini adalah
  untuk meningkatkan toleransi jaringan terhadap kemungkinan terjadinya
  kegagalan mesin, dan juga meningkatkan fleksibelitas jaringan untuk
  berkembang untuk memenuhi kebutuhan.

  Router dalam topologi ini dihubungkan dengan memanfaatkan static routing,
  dengan tujuan agar menambah jalur alternatif bagi data yang ditransmisikan 
  jika terjadi kegagalan pada salah satu kabel yang menghubungkan router-router
  tersebut.

  Konfigurasi IP yang digunakan dalam menghubungkan router-router akan
  menggunakan traffic 30, karena dalam topologi ini, kabel yang menghubungkan
  dua buah router hanya akan membutuhkan dua buah host id yang berada di
  network yang sama.

  \subsection{Analisis Kebutuhan}

  Perangkat-perangkat yang digunakan dalam simulasi ini terdapat sebagai
  berikut.

  \begin{enumerate}

    \item Router Cisco (Seri 3700-an)

      Karena simulasi ini tidak akan menggunakan fitur ataupun protocol khusus,
      maka router Cisco seri 3700-an akan digunakan.

    \item Unmanaged Switch GNS3

      Switch yang digunakan adalah switch default GNS3, karena di simulasi
      ini switch hanya bertindak sebagai konsentrator.

    \item VPCS (Virtual PC)

      VPCS digunakan sebagai representasi perangkat user yang terhubung pada
      jaringan, tugas nya hanya satu yaitu untuk mesimulasikan aktifitas user.

  \end{enumerate}

  \subsection{Konfigurasi Jaringan}

  Konfigurasi topologi ini akan melewati beberapa tahapan, yaitu pertama
  me-konfigurasi IP dan kemudian mengkonfigurasi routing static.

  \subsubsection{Konfigurasi IP Router}

  \subsubsection{Konfigurasi IP VPCS}

  %---------------------------------------------------------------------------
  \newpage
  %---------------------------------------------------------------------------

  \addSection{BAB III}{PENGUJIAN}

  \subsection{Metode Pengujian}

  Metode pengujian dalam topologi ini akan menggunakan pengujian \emph{Ping}
  yang terbagi kedalam dua kondisi, yaitu kondisi normal dan juga akan diuji
  di kondisi dimana terdapat satu kabel diantara dua router yang akan diputus,
  untuk mesimulasikan sebuah gangguan fisik.

  \subsection{Pengujian Ping}

  %---------------------------------------------------------------------------
  \newpage
  %---------------------------------------------------------------------------

  \addSection{BAB IV}{KESIMPULAN}

  %---------------------------------------------------------------------------
  \newpage
  %---------------------------------------------------------------------------

  \pagenumbering{gobble}

  \begin{center}
    \section*{DAFTAR PUSTAKA}
  \end{center}
  \addcontentsline{toc}{section}{DAFTAR PUSTAKA}{}

  \vspace{1cm}

  \begin{enumerate}
    \item foo
  \end{enumerate}

\end{document}
