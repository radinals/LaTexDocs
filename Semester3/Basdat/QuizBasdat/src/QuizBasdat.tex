\documentclass[10pt,a4paper]{article}

\usepackage{geometry}
\usepackage{graphicx}
\usepackage{ragged2e}
\usepackage{multicol}

\usepackage{listings}
\usepackage{xcolor}

\lstset{
  basicstyle=\ttfamily\normalfont\scriptsize,
  breakatwhitespace=true,         
  escapeinside={\%*}{*)},
  breakautoindent=true,
  breaklines=true,                 
  captionpos=b,                    
  keepspaces=false,                 
  showspaces=false,                
  showstringspaces=false,
  showtabs=false,                  
  frame=single,
  numbers=none,
  stepnumber=1,% the step between two line-numbers. If it's 1 each line will be numbered
  tabsize=2
}

% Bash
\lstdefinelanguage{Bash}{
  morekeywords={
    if,then,else,elif,fi,for,while,do,done,case,esac,export
  },
  sensitive=true,
  morecomment=[l]\#,
  morestring=[b]",
}

% C
\lstdefinelanguage{C}{
  morekeywords={
    auto,break,case,char,const,continue,default,do,double,else,
    enum,extern,for,if,int,long,register,return,switch,typedef,
    unsigned,void,volatile,while
  },
  sensitive=true,
  morecomment=[l]//,
  morecomment=[s]/* */ ,
  morestring=[b]",
}

% C++
\lstdefinelanguage{C++}{
  morekeywords={
    alignas,alignof,and,asm,auto,bitand,bitor,bool,break,case,
    class,compl,const,constexpr,continue,decltype,default,delete,
    do,double,else,enum,explicit,false,for,friend,goto,if,inline,
    int,long,mutable namespace,new noexcept,operator,private,
    protected,public,register,return,short,signed,sizeof,
    static,static_assert,static_cast,struct,switch,template,this,
    thread_local,throw,true,try,typedef,typeid,typename,union,
    unsigned,using,virtual,void,volatile,wchar_t,while
  },
  sensitive=true,
  morecomment=[l]//,
  morecomment=[s]/* */ ,
  morestring=[b]",
}

% Java
\lstdefinelanguage{Java}{
  morekeywords={
    abstract,assert,boolean,break,byte,case,catch,char,class,
    const,continue,default,do,double,else,enum,extends,final,
    finally,float, for,goto,if,implements,import,instanceof,int,
    interface,long,native,new, null,package,private,protected,
    public,return,short,static,strictfp, super,switch,
    synchronized,this,throw,throws,transient,true,try,void,
    volatile,while
  },
  sensitive=true,
  morestring=[b]",
}

% Go
\lstdefinelanguage{Go}{
  morekeywords={
    break,case,chan,const,continue,
    default,defer,else,fallthrough,
    for,function,goto,if,import,interface,
    map,package,range,return,select,struct,
    switch,type,var},
  sensitive=true,
  morecomment=[l]//,
  morecomment=[s]/* */ ,
  morestring=[b]",
}

% PHP
\lstdefinelanguage{PHP}{
  morekeywords={
    __halt_compiler,abstract,alias,arguments,break,case,class,
    clone,const,continue,declare,default,die,do,echo,else,elseif,
    empty,endswitch,eval,exit,extends,final,finally,for,foreach,
    function,global,goto,if,implements,include,include_once,
    instanceof,insteadof,interface,is,isset,list,namespace,
    print,private,protected,public,return,static,switch,throw,
    trait,try,unset,use,var,while,yield
  },
  sensitive=true,
  morecomment=[l]//,
  morecomment=[s]/* */ ,
  morestring=[b]",
}

% Javascript
\lstdefinelanguage{JavaScript}{
    morekeywords={abstract,arguments,await,boolean,break,byte,
    case,catch,class,const,continue,debugger,default,delete,do,
    double,else,enum,eval,export,extends,false,finally,for,function,
    global,if,implements,import,in,instanceof,int,let,match,namespace,
    NaN,private,protected,public,return,super,switch,throw,throws,true,
    try,typeof,var,void,yield
  },
  sensitive=true,
  morecomment=[l]//,
  morecomment=[s]/* */ ,
  morestring=[b]",
}



\geometry{margin=1cm}
\graphicspath { {./img/} }

\pagenumbering{gobble}
\date{}

\newcommand{\codeListing}[3] {
  \lstinputlisting{./code/soal#1.sql}
}

\lstdefinelanguage{SQL}{
  keywords={
    select, from, where, group, by, order, asc, desc, and, or, sum, count,
    as, is, null, not, like, join, on, inner, left, right, outer, distinct
  },
  keywordstyle=\bfseries\color{blue}, % Keywords in bold blue
  identifierstyle=\color{black},     % Identifiers in black
  stringstyle=\color{teal},          % Strings in teal
  commentstyle=\itshape\color{gray}, % Comments in italic gray
  morecomment=[l][\itshape\color{gray}]{--}, % SQL-style comments with --
  morestring=[b]"                   % Double-quoted strings
}

\lstset{
  language=SQL,
  basicstyle=\ttfamily\footnotesize, % Basic font style for all text
  numbers=none,                      % Line numbers on the left
  stepnumber=0,                      % Numbering each line
  numbersep=4pt,                     % Space between number and text
  backgroundcolor=\color{white},     % Background color
  frame=single,                      % Box around the code
  captionpos=b,                      % Caption at the bottom
  breaklines=true,                   % Automatic line breaking
  breakatwhitespace=true,            % Break lines only at whitespace
  showspaces=false,                  % Don't show spaces
  showstringspaces=false,            % Don't show spaces in strings
}


\begin{document}

  NAMA: Radinal Shidiq Saragih

  KELAS: IF C 2023

  NPM: 5520123104

  \begin{enumerate}
    \item Perusahan yang bergerak di bidang pariwisata akan membuat system
          informasi pengelolaan VILA yang tersebar di wilayah cianjur dan puncak
          bogor. Dengan permintaan pelanggan harus melakukan login terhadap aplikasi
          dan memilih berbagai type Vila dan fasilitas vila yang ada. Kemudian
          pelanggan akan melakukan booking atas vila tersebut dengan menyertakan
          berapa jumlah penghuni villa dewasa dan anak anak. Perusahaan tersebut
          juga mempunyai karyawan yang harus login pada aplikasi/system tersebut
          sesuai dengan rolenya masing-masing. Selain itu juga dalam vila tersebut
          mempunyai restoran dengan berbagai macam menu makanan yang bisa di pesan
          baik di anter ke villa maupun pesan langsung ke restoran, dan pelanggan
          memalkukan pemayaran atas pesanan makanan dengan pesanan vila, dari kasus
          ini buatlah

      \begin{itemize}

        \item ERD

          \includegraphics[scale=0.5]{"erd.png"}

        \item Physical Data Model

          \includegraphics[scale=0.4]{"physicalmodel.png"}

      \end{itemize}

    \newpage

    \item Tampilkan data NPM,NAMAMHS,ANGKATAN,PRODI,ProgramStudi,
          Sts\_Pendidikan,Deskripsi yang berelasi pada table tmhs,tprodi dan
          tstauskelas denga out put sebagai berikut gunakan fungsi Right


          \codeListing{1}


    \item Tampilkan jumlah masiswa per program study table join relasi Tmhs
          dan tprodi


          \codeListing{2}


    \item Tampikan data dari beban SKS join relasi ke tprodi dan tstauskelas
          dengan output seperti di bawah ini


          \codeListing{3}


    \item Tampilan jumlah pendapatan per prodi di ambil dari
          table tunsur\_payment join table dengan tprodi dengan output
          sebegai berikut


          \codeListing{4}


    \item Tampilkan nilai dengan output sebagai berikut join relasi tnilai,
          tmhs dan tprodi


          \codeListing{5}


    \item Tampilkan mahasiswa yang Alamat nya di cipetir


          \codeListing{6}


    \item Tampikan mahasiswa yang tempat lahir nya selain di cianjur


          \codeListing{7}


    \item Tampilkan mahasiwa yang umur nya di bawah sama dengan 20 tahun


          \codeListing{8}


    \item Tampilkan jumlah dosen berdasarkan jenis kelamin


          \codeListing{9}


    \item Tampilkan data mahasiswa


          \codeListing{10}


  \end{enumerate}

\end{document}
