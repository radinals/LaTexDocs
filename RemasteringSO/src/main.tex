\documentclass[8pt, a4paper]{article}
\usepackage{multicol}
\usepackage{listings}
\usepackage{geometry}
\usepackage{graphicx}
\usepackage{caption}
\usepackage{xstring}
\usepackage{verbatim}


\geometry{margin=0.5cm}
\graphicspath{{img}}

\pagenumbering{gobble}
\date{}

\lstdefinestyle{customstyle}{
    basicstyle=\ttfamily\scriptsize,
    breakatwhitespace=true,         
    breakautoindent=true,
    breaklines=true,                 
    captionpos=t,                    
    keepspaces=false,                 
    showspaces=false,                
    showstringspaces=false,
    showtabs=false,                  
    frame=none,
    stepnumber=1,                                           
    tabsize=2
}

\lstset{style=customstyle}

\begin{document}

\section{Konfigurasi}

\subsection{Tema GTK}

GTK, adalah library GUI yang umum digunakan di linux. Umumnya konfigurasi untuk
merubah warna atau tema yang digunakan oleh aplikasi yang menggunakan toolkit GTK
dilakukan pada sebuah file yang biasanya dituliskan dalam syntax css. File stylesheet
atau GTK dapat berjumlah lebih dari satu, karena untuk tiap versi GTK akan dibedakan
dari stylesheet mana tema akan dibaca.

Karena stylesheet gtk dapat berjumlah banyak umumnya stylesheet-stylesheet
tersebut akan dikumpulkan kedalam satu folder seperti sebagai berikut.

\begin{itemize}
  \item theme/
  \begin{itemize}
    \item gtk-3.0/
    \begin{itemize}
      \item gtk.css
      \item gtk-dark.css 
    \end{itemize}
    \item gtk-4.0/
    \begin{itemize}
      \item gtk.css
      \item gtk-dark.css 
    \end{itemize}
  \end{itemize}
\end{itemize}

disini ada dua versi gtk yang dapat dikonfigurasi, yaitu 3.0 dan 4.0, dari
kedua versi ini yang paling banyak sudah diadopsi adalah versi 3.0, karena
versi 4.0 tergolong baru dikenalkan, jadi lebih banyak aplikasi yang menggunakan
versi 3.0.

Dalam guna remastering, konfigurasi dilakukan pada css gtk versi 3.0, dengan
merubah beberapa class dan menyesuaikannya kedalam tema yang telah dirancang.

Berikut adalah contoh konfigurasi GTK untuk merubah tampilan panel dan tema XFCE.

\begin{multicols}{2}
\begin{lstlisting}

@define-color BlueDarker_Custom #102444;
@define-color BlueDark_Custom #1a3059;
@define-color Blue_Custom #203e70;
@define-color BlueLightLighter_Custom #2e64bd;
@define-color BlueLight_Custom #92a8c0;
@define-color White_Custom #eef3ed;

/* XFCE4 PANEL */
.xfce4-panel.background {
  background: @Blue_Custom;
}

.xfce4-panel button {
  border: 0;
  border-radius: 0;
  min-width: 30px;
  box-shadow: none;
  text-shadow: none;
  background: inherit;
  color: #eef3ed; }

  .xfce4-panel button:hover {
    background: @Blue_Custom;
  }

  .xfce4-panel button:not(#whiskermenu-button):checked,
  .xfce4-panel button:not(#whiskermenu-button):active
  {
    border-radius: 4px;
    border: 1px solid @BlueDarker_Custom;
    color: #000000;
    background: rgba(238.0, 243.0, 237.0, 0.8);
  }

  .xfce4-panel.vertical button:not(#whiskermenu-button):checked,
  .xfce4-panel.vertical button:not(#whiskermenu-button):active 
  {
    box-shadow: inset 2px 0 0 @BlueDarker_Custom; 
  }

/* XFCE4 NOTIFICATIONS & LOGOUT DIALOG */
#XfceNotifyWindow, .xfsm-logout-dialog, .xfsm-logout-dialog button {
  border: 1px solid #d9d9d9;
  border-bottom-width: 2px;
  box-shadow: inset 0 0 0 1px rgba(255, 255, 255, 0.1);
  -gtk-icon-style: symbolic; }

.xfsm-logout-dialog > .vertical > .horizontal button {
  margin: 8px; }
\end{lstlisting}
\end{multicols}

\subsection{Konfigurasi LightDM/Lock Screen}

Konfigurasi pada lockscreen di Xubuntu dapat dilakukan dengan merubahnya di konfigurasi
css gtk selayaknya tema yang sebelumnya dijelaskan, potongan konfigurasi dapat
dilihat sebagai berikut.

\begin{multicols}{2}
\begin{lstlisting}
.lightdm-gtk-greeter #panel_window {
  background: rgba(32.0, 62.0, 112.0, 1);
  color: @White_Custom;
  text-shadow: none;
  -gtk-icon-shadow: none; }

.lightdm-gtk-greeter #login_window {
  border: 0;
  background-color: rgba(32.0, 62.0, 112.0, 0.8);
  color: @White_Custom;
  border-radius: 4px;
  box-shadow: inset 0 1px rgba(255, 255, 255, 0.15); }
  .lightdm-gtk-greeter #login_window entry {
    background: @White_Custom;
    border-color: rgba(255, 255, 255, 0.3);
    min-height: 32px;
    box-shadow: inset 0 1px rgba(255, 255, 255, 0.05), 0 1px 2px rgba(0, 0, 0, 0.2); }
    .lightdm-gtk-greeter #login_window entry image {
      margin: 0;
   }
  .lightdm-gtk-greeter #login_window #buttonbox_frame {
    padding-top: 20px;
    background: rgba(32.0, 62.0, 112.0, 1);
    border-top: 1px solid rgba(0, 0, 0, 0.1);
    /*border-radius: 0 0 4px 4px; */
    box-shadow: inset 0 -4px rgba(0, 0, 0, 0.2), inset 0 1px 2px rgba(0, 0, 0, 0.07); }
    .lightdm-gtk-greeter #login_window #buttonbox_frame button {
      border: 0;
      color: #000000;
      text-shadow: none; }
    .lightdm-gtk-greeter #login_window #buttonbox_frame #cancel_button {
      background: @White_Custom;
      box-shadow: none;
      color: #000000; }
    .lightdm-gtk-greeter #login_window #buttonbox_frame #login_button {
      background: @White_Custom;
      box-shadow: inset 0 -2px rgba(0, 0, 0, 0.2); }
\end{lstlisting}
\end{multicols}

\newpage

\subsection{Konfigurasi Splash Screen/Loading Screen}

Konfigurasi pada splash screen dapat dilakukan dengan membuat sebuah tema
plymouth, plymouth adalah software yang nantinya di saat boot dan shutdown
akan dijalankan untuk menampilkan layar-layar loading.

Bentuk dari sebuah tema plymouth sederhana terdiri atas,

\begin{enumerate}
  \item file .plymouth
    Berisikan informasi tentang module-module dan tema yang ingin dibuat

  \item file .script
    Berisikan logika dari tema, kapan, bagaimana atau dimana gambar ditampilkan.

  \item asset atau grafik
    asset adalah file-file yang akan digunakan dalam script agar dapat menampilan
    gambar.

\end{enumerate}

Konfigurasi dilakukan dengan membuat script yang akan menampilkan asset-asset yang
ingin ditampilkan. Dan untuk menerapkan nya pada system dapat dilakukan melalui
terminal untuk mengatur tema mana yang akan digunakan.

\end{document}
