\documentclass[12pt, a4paper]{book}

\usepackage[T1]{fontenc}
\usepackage{courier}
\usepackage{graphicx}
\usepackage{mathptmx}
\usepackage{ragged2e}
\usepackage{xcolor}
\usepackage{listings}
\usepackage{enumitem}

\graphicspath{{../img}}
\pagenumbering{gobble}

\definecolor{mGreen}{rgb}{0,0.6,0}
\definecolor{mGray}{rgb}{0.5,0.5,0.5}
\definecolor{mPurple}{rgb}{0.58,0,0.82}
\definecolor{backgroundColour}{rgb}{0.95,0.95,0.92}

\lstset{
  language=C,                % choose the language of the code
  % numbers=left,                   % where to put the line-numbers
  stepnumber=1,                   % the step between two line-numbers.
  numbersep=1pt,                  % how far the line-numbers are from the code
  backgroundcolor=\color{backgroundColour},  % choose the background color. You must add \usepackage{color}
  showspaces=false,               % show spaces adding particular underscores
  showstringspaces=false,         % underline spaces within strings
  showtabs=false,                 % show tabs within strings adding particular underscores
  tabsize=2,                      % sets default tabsize to 2 spaces
  captionpos=b,                   % sets the caption-position to bottom
  breaklines=true,                % sets automatic line breaking
  breakatwhitespace=true,         % sets if automatic breaks should only happen at whitespace
  basicstyle=\footnotesize\ttfamily
  % title=\lstname,                 % show the filename of files included with \lstinputlisting;
}

\begin{document}
     \begin{center}
         \vspace*{1cm}

         \textbf{\Huge{Design Patterns}}

         \vspace{0.2cm}

          C++ Design Patterns Study Notes

          By: RSS

         \vspace{0.2cm}

     \end{center}

     \newpage{}
  \pagenumbering{arabic}

  \chapter {OOP Concepts}

   \begin{flushright}
   ``Object-oriented design is the roman numerals of computing.``

   Rob Pike
   \end{flushright}

  \section{Polymorphism}

    Single symbol to represent multiple types. A Symbol can behave in several
    different forms.

    Polymorphism enables many objects to behave the same way, a common interface.

    Polymorphism can be devided to two types, Run-Time Polymorphism,
    and Compile-Time Polymorphism. Run-Time Polymorphism, or dynamic or late
    binding, is polymorphism that is resolved in the processes lifetime. Compile-Time
    Polymorphism, or static or early binding, is polymorphism that
    is resolved in the compilation process.

    There are also a few common forms of polymorphism, as stated below.

    \begin{enumerate}[label=\arabic*.]

      \item \textbf{Ad Hoc}

        Common Interface for a arbitrary set of individually specified types.

        \lstinputlisting{cpp_source/PolymorphismFormsExample/adHocObj.cpp}

      \item \textbf{Parametric}

        Using Abstract types instead of using concrete types that can be
        subtitute for any type.

        \lstinputlisting{cpp_source/PolymorphismFormsExample/parametricPoly.cpp}

      \item \textbf{Subtyping/Inclusion}

        A name that denotes instances of multiple different classes related
        by some common superclass.

        \lstinputlisting{cpp_source/PolymorphismFormsExample/subtyping.cpp}

    \end{enumerate}

  \section{Inheritance}

    A way of deriving a new class and also inherits traits from the earlier
    existing class this is the fundemental idea behind reusable code in OOP.

    There is several types of Inheritance, as given below.

    \begin{enumerate}[label=\arabic*.]
      \item Single Inheritance

        One Sub class inherits from one Super class.

        \lstinputlisting{cpp_source/InheritanceExample/SingleInheritance.cpp}

      \item Multiple Inheritance

        One Sub class inherits from two or more Super class.

        \lstinputlisting{cpp_source/InheritanceExample/MultipleInheritance.cpp}

      \item Multilevel Inheritance

        One Sub class inherits from another Sub class of a Super class

        \lstinputlisting{cpp_source/InheritanceExample/MultilevelInheritance.cpp}

      \item Hierarchical inheritance
      \item Hybrid inheritance
    \end{enumerate}

  \section{Encapsulation}

  Hiding complexity from the implementation inside a class.

  \section{Abstraction}

  Simplifying features by hiding the complicated/unimportant details.

  \section{Data Hiding}

  Hiding data inside a class from the user (programmer).

  \section{Message Passing}

  Communication between objects.

\end{document}
