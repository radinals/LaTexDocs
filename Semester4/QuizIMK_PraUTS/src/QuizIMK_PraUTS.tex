\documentclass[12pt,a4paper]{article}

\usepackage{geometry}
\usepackage{graphicx}
\usepackage{ragged2e}
\usepackage{multicol}

\usepackage{listings}
\usepackage{xcolor}

\lstset{
  basicstyle=\ttfamily\normalfont\scriptsize,
  breakatwhitespace=true,         
  escapeinside={\%*}{*)},
  breakautoindent=true,
  breaklines=true,                 
  captionpos=b,                    
  keepspaces=false,                 
  showspaces=false,                
  showstringspaces=false,
  showtabs=false,                  
  frame=single,
  numbers=none,
  stepnumber=1,% the step between two line-numbers. If it's 1 each line will be numbered
  tabsize=2
}


% Bash
\lstdefinelanguage{Bash}{
  morekeywords={
    if,then,else,elif,fi,for,while,do,done,case,esac,export
  },
  sensitive=true,
  morecomment=[l]\#,
  morestring=[b]",
}

% C
\lstdefinelanguage{C}{
  morekeywords={
    auto,break,case,char,const,continue,default,do,double,else,
    enum,extern,for,if,int,long,register,return,switch,typedef,
    unsigned,void,volatile,while
  },
  sensitive=true,
  morecomment=[l]//,
  morecomment=[s]/* */ ,
  morestring=[b]",
}

% C++
\lstdefinelanguage{C++}{
  morekeywords={
    alignas,alignof,and,asm,auto,bitand,bitor,bool,break,case,
    class,compl,const,constexpr,continue,decltype,default,delete,
    do,double,else,enum,explicit,false,for,friend,goto,if,inline,
    int,long,mutable namespace,new noexcept,operator,private,
    protected,public,register,return,short,signed,sizeof,
    static,static_assert,static_cast,struct,switch,template,this,
    thread_local,throw,true,try,typedef,typeid,typename,union,
    unsigned,using,virtual,void,volatile,wchar_t,while
  },
  sensitive=true,
  morecomment=[l]//,
  morecomment=[s]/* */ ,
  morestring=[b]",
}

% Java
\lstdefinelanguage{Java}{
  morekeywords={
    abstract,assert,boolean,break,byte,case,catch,char,class,
    const,continue,default,do,double,else,enum,extends,final,
    finally,float, for,goto,if,implements,import,instanceof,int,
    interface,long,native,new, null,package,private,protected,
    public,return,short,static,strictfp, super,switch,
    synchronized,this,throw,throws,transient,true,try,void,
    volatile,while
  },
  sensitive=true,
  morecomment=[l]//,
  morecomment=[s]/* */ ,
  morestring=[b]",
}

% Go
\lstdefinelanguage{Go}{
  morekeywords={
    break,case,chan,const,continue,
    default,defer,else,fallthrough,
    for,function,goto,if,import,interface,
    map,package,range,return,select,struct,
    switch,type,var},
  sensitive=true,
  morecomment=[l]//,
  morecomment=[s]/* */ ,
  morestring=[b]",
}

% PHP
\lstdefinelanguage{PHP}{
  morekeywords={
    __halt_compiler,abstract,alias,arguments,break,case,class,
    clone,const,continue,declare,default,die,do,echo,else,elseif,
    empty,endswitch,eval,exit,extends,final,finally,for,foreach,
    function,global,goto,if,implements,include,include_once,
    instanceof,insteadof,interface,is,isset,list,namespace,
    print,private,protected,public,return,static,switch,throw,
    trait,try,unset,use,var,while,yield
  },
  sensitive=true,
  morecomment=[l]//,
  morecomment=[s]/* */ ,
  morestring=[b]",
}

% Javascript
\lstdefinelanguage{JavaScript}{
    morekeywords={abstract,arguments,await,boolean,break,byte,
    case,catch,class,const,continue,debugger,default,delete,do,
    double,else,enum,eval,export,extends,false,finally,for,function,
    global,if,implements,import,in,instanceof,int,let,match,namespace,
    NaN,private,protected,public,return,super,switch,throw,throws,true,
    try,typeof,var,void,yield
  },
  sensitive=true,
  morecomment=[l]//,
  morecomment=[s]/* */ ,
  morestring=[b]",
}



\geometry{margin=1cm}
\graphicspath { {./img/} }

\pagenumbering{gobble}
\date{}

\begin{document}
Radinal Shidiq Saragih

IF C 2023

5520123104

\begin{center}
  \section*{QUIZ PRA UAS IMK}
  \vspace{1cm}
\end{center}

\begin{enumerate}
  \item Ragam Dialog dalam pembuatan Interface memiliki beberapa sifat salah satunya memiliki sifat keluwesan jelaskan maksud dari sifat keluwesan tersebut!

    Keluwesan dalam interface mengacu pada kemampuan sebuah interface atau antarmuka untuk
    beradaptasi dengan kebutuhan pengguna yang beragam. Interface yang luwes membuat
    pengalaman user yang intuitif, inklusif dan efisien. Intuitif dengan maksud dalam
    penggunaan interface tersebut memiliki fleksibilitas dalam berinteraksi dengan
    interface hingga dapat digunakan dengan mudah atau disebut juga user-friendly.
    Inklusif karena dalam interface yang luwes, interface dapat beradaptasi sesuai
    kebutuhan penggunanya jadi siapapun dapat menggunakan. Dan efisien karena
    dapat membantu pengguna menyelesaikan pekerjaan dengan secepat mungkin dan
    tidak melambatkan pekerjaan pengguna dengan interface.

  \item Sebutkan Petunjuk dari aspek kognitif dari penggunaan warna dalam pembuatan interface!

    Dalam perancangan sebuah interface terdapat berbagai petunjuk-petunjuk kognitif yang dapat
    digunakan dalam interface untuk memastikan interface yang dirancang dapat mudah
    dipahami penggunanya. Antara lain.

    - Tingkatan Kontras dan Hierarki Warna

    Perbedaan kontras warna untuk elemen intrface untuk meng-highlight bagian yang
    penting.

    Warna juga dapat digunakan untuk menandai status atau suatu keadaan, semisalnya
    hijau jika berhasil dan merah jika sesuatu gagal dilakukan.

    - Konsistensi Warna

    Perancangan elemen dengan menggunakan warna yang konsisten dapat memberi
    kesan pengelompokan terhadap elemen tersebut.

    - Psikologi Warna

    Pemilihan warna dengan memerhatikan makna psikologis dengan harapan untuk
    memicu respon emosional user.

  \item Ada beberapa ilmu yang mendukung untuk pembuatan antarmuka yang ramah
    terhadap pengguna salah satunya ilmu sosiologi jelaskan manfaat dari ilmu sosiologi tersebut

    Psikologi termasuk ilmu yang berkaitan dengan perancangan antarmuka karena
    Psikologi sebagai ilmu yang mempelajari tentang pola pikir dan emosional
    manusia, yang juga sekaligus adalah target pengguna dari tiap rancangan
    antaramuka yang pernah dibuat. Tanpa pengetahuan psikologis terdapat kesenjangan
    antara antarmuka yang dibuat dengan target pengguna, karena kita tidak dapat
    memahami karakter dari si pengguna tersebut.

  \item Di dalam membuat sebuah Implementasi ada tiga prinsip yang harus
    dilakukan sebutkan ketiga prinsip tersebut?

    1. Efektivitas

    Implementasi harus mampu mencapai tujuan yang telah dirancang.
    Setiap fitur atau proses harus benar-benar memberikan manfaat sesuai dengan kebutuhan pengguna atau sistem.
    Contoh: Dalam implementasi aplikasi, sistem login harus bekerja dengan baik tanpa mengalami kesalahan akses.

    2. Efisiensi

    Implementasi harus dilakukan dengan pemanfaatan sumber daya yang optimal, baik dari segi waktu, tenaga, maupun biaya.
    Meminimalkan redundansi dan memastikan performa tetap stabil.
    Contoh: Dalam pemrograman, penggunaan algoritma yang lebih cepat dan hemat memori untuk meningkatkan kinerja sistem.

    3. Keberlanjutan (Sustainability)

    Implementasi harus dapat terus berjalan dalam jangka panjang dan mudah untuk dikembangkan lebih lanjut.
    Memastikan sistem yang dibuat mudah diperbarui, diperbaiki, dan diadaptasi terhadap perubahan di masa depan.
    Contoh: Penggunaan teknologi yang fleksibel dan dokumentasi yang baik agar sistem bisa dikembangkan tanpa kesulitan.


  \item Membuat Interface ada tiga tahap salah satunya adalah Mockup jelaskan
    bagaimana membuat sebuah mockup untuk interface Log in?

    1. Analisis Kebutuhan

    Langkah pertama yang perlu dilakukan adalah menentukan jenis login yang
    dibutuhkan oleh sistem. Seperti apakah dibutuhkan opsi untuk merubah
    password atau membuat akun, atau cukup dengan login saja dan lain 
    sebagainya. Di tahapan ini juga ditentukan data apa yang diperlukan untuk
    login.

    2. Menyusun Tata Letak Elemen Login

    Setelah kebutuhan rancangan ditentukan maka selanjutnya dapat dimulai
    penyusunan elemen-elemen login, seperti letak tombol dan field-field input.
    Penyusunan didokumentasikan dalam sebuah wireframe.

    3. Pembuatan mockup

    Setelah wireframe dibentuk maka selanjutnya sebuah bentuk mockup dapat
    mulai dibuat dan dengan beracuan wireframe untuk peletakan, di mockup
    warna-warna dan petunjuk-petunjuk visual dapat ditambahkan dan dirumuskan.
    hingga menyerupai tampilan yang diinginkan.

    4. Revisi dan Perbaikan

    Setelah mockup dibuat maka selanjutnya adalah kembali menganalisis kesesuaian
    design dengan kebutuhan sistem, apakah sesuai dengan requirement sistem atau
    tidak. Dan jika timbul ketidaksesuaian dapat direvisi hingga mendapatkan
    mockup yang disetujui dan sesuai.

\end{enumerate}
\end{document}
