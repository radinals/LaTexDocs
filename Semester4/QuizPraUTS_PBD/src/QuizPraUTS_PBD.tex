\documentclass[a4paper,11pt]{article}
\usepackage[utf8]{inputenc}
\usepackage{enumitem}
\usepackage{geometry}
\geometry{left=2.5cm, right=2.5cm, top=2.5cm, bottom=2.5cm}

\date{}
\pagenumbering{gobble}
\begin{document}

Radinal Shidiq Saragih

IF C 2023

5520123104

\section*{1. Kategori Media Penyimpanan}
\begin{enumerate}[label=\alph*.]
    \item Jelaskan perbedaan antara penyimpanan volatile dan non-volatile serta berikan masing-masing contohnya!

      Penyimpanan volatile adalah jenis penyimpanan yang membutuhkan daya
      listrik untuk mempertahankan data. Contohnya adalah RAM.

      Penyimpanan non-volatile tidak membutuhkan daya untuk mempertahankan
      data, seperti hard disk (HDD) dan solid-state drive (SSD).

      Sehingga jika disuatu saat mesin dimatikan penyimpanan volatile akan
      kehilangan semua data yang disimpan sedangkan penyimpanan non-volatile
      akan tetap mempertahankan data yang disimpan didalamnya.

    \item Sebutkan dan jelaskan tiga jenis penyimpanan non-volatile yang umum digunakan dalam sistem basis data!

      \begin{itemize}
        \item Hard Disk Drive (HDD)

          Media penyimpanan magnetik yang digunakan secara luas dalam sistem basis data karena kapasitas besar dan biaya rendah.

        \item Solid State Drive (SSD)

          Penyimpanan berbasis flash yang lebih cepat dari HDD dengan akses waktu rendah.

        \item Tape Storage 

          Digunakan untuk penyimpanan jangka panjang dan pencadangan karena kapasitas besar dan biaya per GB yang rendah.

      \end{itemize}


    \item Mengapa memori utama (RAM) tidak digunakan sebagai satu-satunya media penyimpanan dalam sistem basis data?

      Memori utama tidak digunakan karena hanya bersifat volatile, hingga akan
      kehilangan data jika daya listrik tidak ada. Lalu memori utama juga umumnya
      berukuran sangat kecil dibanding penyimpanan yang diberi penyimpanan sekunder,
      maka akan terbatas dengan jumlah data yang dapat disimpan, dan untuk memperbanyak
      ukuran dari memori utama hingga mengimbangi ukuran penyimpanan sekunder 
      jauh lebih mahal daripada membeli HDD atau SSD.

\end{enumerate}

\section*{2. Struktur Media Penyimpanan}
\begin{enumerate}[label=\alph*.]
    \item Jelaskan konsep hierarki penyimpanan dalam sistem basis data dan sebutkan level-levelnya!


      \begin{itemize}
        \item Cache

          Penyimpanan tercepat dengan kapasitas kecil. Digunakan untuk mempercepat
          akses data yang sering dibuka di media penyimpanan yang lambat.

        \item RAM

          Digunakan untuk menyimpan data sementara selama eksekusi.
          
        \item Penyimpanan Sekunder

          Penyimpanan utama yang berukuran besar untuk data basis data.

        \item Penyimpanan Tersier

          Digunakan untuk pencadangan dan arsip atau backup.

        \end{itemize}

    \item Bandingkan hard disk drive (HDD) dan solid state drive (SSD) dalam konteks penggunaan basis data!

      Dari segi kecepatan SSD memiliki kecepatan read-write yang lebih baik, karena tidak bergantung
      pada komponen-komponen bergerak seperti piringan di HDD. Namun HDD memiliki harga
      yang lebih murah dibanding SSD, jadi HDD berkapasitas besar dapat dibeli dengan harga jauh
      lebih murah dibanding ukuran yang sama dibentuk HDD. Lalu ada juga faktor kerusakan,
      HDD lebih beresiko mengalami kerusakan dibanding SSD karena memiliki lebih banyak komponen
      bergerak.

    \item Bagaimana cara kerja mekanisme buffering dalam basis data untuk meningkatkan efisiensi penyimpanan?

      Mekanisme Buffering adalah mekanisme untuk meningkatkan efisiensi penyimpanan dengan menyimpan
      secara sementara data yang sering diakses ke memori sebelum ditulis ke penyimpanan sekunder.
      Tujuannya agar data yang berulang kali diakses dapat dengan cepat diakses untuk kesempatan
      berikutnya karena tidak harus mencari langsung ke disk.

\end{enumerate}

\section*{3. Konsep Indexing dalam Basis Data}
\begin{enumerate}[label=\alph*.]
    \item Apa yang dimaksud dengan indexing dalam basis data dan mengapa hal ini penting?

      Indexing dalam basis data adalah metode untuk meningkatkan kecepatan
      pencarian data dengan menggunakan struktur indeks yang menunjuk
      ke lokasi data dalam tabel. Indexing biasanya dilakukan pada data-data yang
      sering diakses.

    \item Sebutkan dan jelaskan perbedaan antara Primary Index dan Secondary Index!

    \begin{itemize}

        \item Primary Index

          Dibangun pada atribut kunci utama dan bersifat unik.

        \item Secondary Index

          Dibangun pada atribut non-kunci untuk mempercepat pencarian tambahan.

    \end{itemize}

    \item Bagaimana cara kerja B-Tree Index dalam mempercepat pencarian data dalam basis data?

      B-Tree Index bekerja dengan cara menyimpan data dalam struktur pohon
      yang seimbang, memungkinkan pencarian biner yang cepat dengan
      kompleksitas \(O(\log n)\).

\end{enumerate}

\section*{4. Concurrency Control}
\begin{enumerate}[label=\alph*.]
    \item Jelaskan konsep concurrency control dan mengapa diperlukan dalam sistem basis data!

     Concurrency control adalah mekanisme untuk mengelola eksekusi transaksi
     secara bersamaan agar mencegah inkonsistensi data. Concurrency control
     perlu dilakukan karena tanpa pengendalian akses bersamaan pada sebuah data
     dapat menimbulkan inkonsistensi data (Ghost) yang dapat mengakibatkan
     bug-bug yang tersembunyi pada sistem yang bergantung pada data tersebut.

    \item Bagaimana mekanisme Two-Phase Locking (2PL) dapat mencegah terjadinya inkonsistensi data dalam transaksi basis data?

    \begin{itemize}
        \item Growing Phase

          Transaksi memperoleh semua lock yang dibutuhkan.

        \item Shrinking Phase

          Transaksi melepaskan lock setelah selesai.

    \end{itemize}


    \item Dua transaksi berikut berjalan secara bersamaan dalam sebuah sistem basis data yang mengakses tabel \textit{Saldo\_Akun}:
    
    \textbf{Transaksi T1:}
    \begin{itemize}
        \item Read(A)
        \item A = A + 5000
        \item Write(A)
        \item Commit
    \end{itemize}

    
    \textbf{Transaksi T2:}
    \begin{itemize}
        \item Read(A)
        \item A = A * 2
        \item Write(A)
        \item Commit
    \end{itemize}

    Asumsikan nilai awal A = 10.000, dan hitung nilai saldo akhirnya dari nilai (A).

    JAWAB:

    Jika T1 dieksekusi lebih dulu: \( A = (10.000 + 5.000) * 2 = 30.000 \)

    Jika T2 dieksekusi lebih dulu: \( A = (10.000 * 2) + 5.000 = 25.000 \)

\end{enumerate}

\section*{5. Arsitektur Sistem Basis Data}
\begin{enumerate}[label=\alph*.]
    \item Apa perbedaan antara \textit{Centralized Database System} dan \textit{Distributed Database System}?

      Centralized Database System adalah sistem di mana basis data
      dikelola dalam satu lokasi pusat.

      Distributed Database System membagi basis data ke beberapa lokasi yang tidak
      berdekatan yang terhubung dalam jaringan.

    \item Bagaimana arsitektur \textit{Client-Server} bekerja dalam sistem basis data dan apa kelebihannya dibandingkan dengan sistem berbasis file tradisional?

      \begin{itemize}
        \item Memisahkan tugas pemrosesan dan penyimpanan data. 
        \item Mengurangi duplikasi data dan meningkatkan keamanan.
        \item Server dan klien dapat berada dilokasi yang berbeda.
      \end{itemize}


\end{enumerate}

\end{document}
