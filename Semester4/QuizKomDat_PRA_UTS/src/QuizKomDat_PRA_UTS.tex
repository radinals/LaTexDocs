\documentclass[12pt,a4paper]{article}

\usepackage{geometry}
\usepackage{graphicx}
\usepackage{ragged2e}
\usepackage{multicol}

\usepackage{listings}
\usepackage{xcolor}

\lstset{
  basicstyle=\ttfamily\normalfont\scriptsize,
  breakatwhitespace=true,         
  escapeinside={\%*}{*)},
  breakautoindent=true,
  breaklines=true,                 
  captionpos=b,                    
  keepspaces=false,                 
  showspaces=false,                
  showstringspaces=false,
  showtabs=false,                  
  frame=single,
  numbers=none,
  stepnumber=1,% the step between two line-numbers. If it's 1 each line will be numbered
  tabsize=2
}


% Bash
\lstdefinelanguage{Bash}{
  morekeywords={
    if,then,else,elif,fi,for,while,do,done,case,esac,export
  },
  sensitive=true,
  morecomment=[l]\#,
  morestring=[b]",
}

% C
\lstdefinelanguage{C}{
  morekeywords={
    auto,break,case,char,const,continue,default,do,double,else,
    enum,extern,for,if,int,long,register,return,switch,typedef,
    unsigned,void,volatile,while
  },
  sensitive=true,
  morecomment=[l]//,
  morecomment=[s]/* */ ,
  morestring=[b]",
}

% C++
\lstdefinelanguage{C++}{
  morekeywords={
    alignas,alignof,and,asm,auto,bitand,bitor,bool,break,case,
    class,compl,const,constexpr,continue,decltype,default,delete,
    do,double,else,enum,explicit,false,for,friend,goto,if,inline,
    int,long,mutable namespace,new noexcept,operator,private,
    protected,public,register,return,short,signed,sizeof,
    static,static_assert,static_cast,struct,switch,template,this,
    thread_local,throw,true,try,typedef,typeid,typename,union,
    unsigned,using,virtual,void,volatile,wchar_t,while
  },
  sensitive=true,
  morecomment=[l]//,
  morecomment=[s]/* */ ,
  morestring=[b]",
}

% Java
\lstdefinelanguage{Java}{
  morekeywords={
    abstract,assert,boolean,break,byte,case,catch,char,class,
    const,continue,default,do,double,else,enum,extends,final,
    finally,float, for,goto,if,implements,import,instanceof,int,
    interface,long,native,new, null,package,private,protected,
    public,return,short,static,strictfp, super,switch,
    synchronized,this,throw,throws,transient,true,try,void,
    volatile,while
  },
  sensitive=true,
  morecomment=[l]//,
  morecomment=[s]/* */ ,
  morestring=[b]",
}

% Go
\lstdefinelanguage{Go}{
  morekeywords={
    break,case,chan,const,continue,
    default,defer,else,fallthrough,
    for,function,goto,if,import,interface,
    map,package,range,return,select,struct,
    switch,type,var},
  sensitive=true,
  morecomment=[l]//,
  morecomment=[s]/* */ ,
  morestring=[b]",
}

% PHP
\lstdefinelanguage{PHP}{
  morekeywords={
    __halt_compiler,abstract,alias,arguments,break,case,class,
    clone,const,continue,declare,default,die,do,echo,else,elseif,
    empty,endswitch,eval,exit,extends,final,finally,for,foreach,
    function,global,goto,if,implements,include,include_once,
    instanceof,insteadof,interface,is,isset,list,namespace,
    print,private,protected,public,return,static,switch,throw,
    trait,try,unset,use,var,while,yield
  },
  sensitive=true,
  morecomment=[l]//,
  morecomment=[s]/* */ ,
  morestring=[b]",
}

% Javascript
\lstdefinelanguage{JavaScript}{
    morekeywords={abstract,arguments,await,boolean,break,byte,
    case,catch,class,const,continue,debugger,default,delete,do,
    double,else,enum,eval,export,extends,false,finally,for,function,
    global,if,implements,import,in,instanceof,int,let,match,namespace,
    NaN,private,protected,public,return,super,switch,throw,throws,true,
    try,typeof,var,void,yield
  },
  sensitive=true,
  morecomment=[l]//,
  morecomment=[s]/* */ ,
  morestring=[b]",
}



\geometry{margin=1cm}
\graphicspath { {./img/} }

\pagenumbering{gobble}
\date{}

\begin{document}

Radinal Shidiq Saragih

5520123104

IF C 2023

\begin{enumerate}

  \item Konversi waktu dari ukuran 1µ detik ke dalam mili detik, berapa besaran frekuensi yg dihasilkan dalam ukuran Hz?

    p1 mikrodetik (1µ ) = 1 x 10-6  detik

    1 milidetik (ms) = 1 x 10-3 detik

    1µ = 0.001 ms

    f = 1 / 1 x 10-6  

    f = 1 x 10-6  Hz = 1Mhz

  \item Apa yg Anda ketahui mengenai perangkat DTE dan DCE?

  DTE atau Data Terminal Converter adalah instrumen yang mengubah informasi
  menjadi sinyal yang dapat diterima reconverts. DTE adalah unit fungsional
  dari sebuah stasiun data yang berfungsi sebagai sumber data untuk 
  melakukan komunikasi data dengan fungsi kontrol yang sesuai dengan
  protocol link.

   Data Circuit Equipment (DCE) adalah perangkat yang terletak antara Data
   Terminal Equipment dan Data Circuit Transmisi . Data ini juga disebut 
   peralatan komunikasi data dan operator peralatan data. DCE melakukan fungsi
   seperti sinyal konversi, coding , dan garis clocking dan dapat menjadi bagian
   dari peralatan DTE. 

  Meskipun istilah yang paling sering digunakan adalah RS-232 , namun DTE dan
  DCE merupakan standar dari Peralatan Komunikasi Data yang kedua peralatan
  tersebut saling berkomunikasi. 

    Nama peralatan yang menggunakan peralatan standar ini:

    \begin{itemize}
      \item Federal Standard 1037C
      \item MIL-STD-188
      \item RS-232
      \item Beberapa ITU-T standar dalam seri V (terutama V.24 dan V.35) 
    \end{itemize}


  \item Apa pendapat Anda mengenai konsep Komunikasi Data DIGITAL dan ANALOG?

    Menurut saya konsep mengenai data digital dengan analog mengacu pada
    tipe bentuk data, tepatnya bentuk representasi dari data. Data analog 
    bersumber dari fenomena fisik dan bersifat kontinu (Contohnya: suara,
    cahaya, atau gelombang radio). Data digital berasal dari sistem diskrit
    dan representasikan dengan nilai biner (Contohnya data yang dihasilkan
    komputer atau elektronik yang melakukan konversi dari sinyal analog
    menjadi digital )

  \item Jelaskan perbedaan penggunaan Vignere Cipher dengan playfair cipher?

    Enkripsi Vignere Cipher menggunakan tabel vignere yang memiliki
    huruf-huruf asli dengan bentuk terinkripsi nya yang didapatkan dari
    digesernya urutannya. Kunci yang berulang digunakan untuk menentukan
    pergeseran alfabet tersebut.

    Sedangkan Playfair Cipher menggunakan teknik subtitusi digraf yaitu secara 
    berpasangan (dua huruf). dan menggunakan sebuah grid 5x5 yang berisikan
    huruf alfabet sebagai panduan enkripsi.

  \item Berdasarkan pengetahuan Anda, jelaskan manfaat dari pengelolaan keamanan Internet Protokol?


    Pengelolaan Keamanan Internet Protocol karena sangat penting untuk
    memastikan integritas kerahasiaan sistem terjaga.

    Pengolaan keamanan yang tepat memiliki manfaat sebagai berikut.

    \begin{itemize}
      \item Kontrol Akses dengan membatasi akses terhadap beberapa segmen sensitif jaringan.
      \item Kerahasiaan Informasi dengan mengenkripsi data yang ditransmisikan.
      \item Pencegahan Serangan siber seperti DDoS, Spoofing, dan middle-man-attack.
      \item Mencegah penyadapan (Sniffing) atau modifikasi data yang ditransmisikan.
      \item Standarisi Protocol agar sesuai regulasi transmisi.
    \end{itemize}

  
\end{enumerate}

\end{document}
