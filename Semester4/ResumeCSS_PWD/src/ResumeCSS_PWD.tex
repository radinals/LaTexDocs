\documentclass[12pt,a4paper]{article}

\usepackage{geometry}
\usepackage{graphicx}
\usepackage{ragged2e}
\usepackage{multicol}
\usepackage{listings}

\usepackage{listings}
\usepackage{xcolor}

\lstset{
  basicstyle=\ttfamily\normalfont\scriptsize,
  breakatwhitespace=true,         
  escapeinside={\%*}{*)},
  breakautoindent=true,
  breaklines=true,                 
  captionpos=b,                    
  keepspaces=false,                 
  showspaces=false,                
  showstringspaces=false,
  showtabs=false,                  
  frame=single,
  numbers=none,
  stepnumber=1,% the step between two line-numbers. If it's 1 each line will be numbered
  tabsize=2
}

% Bash
\lstdefinelanguage{Bash}{
  morekeywords={
    if,then,else,elif,fi,for,while,do,done,case,esac,export
  },
  sensitive=true,
  morecomment=[l]\#,
  morestring=[b]",
}

% C
\lstdefinelanguage{C}{
  morekeywords={
    auto,break,case,char,const,continue,default,do,double,else,
    enum,extern,for,if,int,long,register,return,switch,typedef,
    unsigned,void,volatile,while
  },
  sensitive=true,
  morecomment=[l]//,
  morecomment=[s]/* */ ,
  morestring=[b]",
}

% C++
\lstdefinelanguage{C++}{
  morekeywords={
    alignas,alignof,and,asm,auto,bitand,bitor,bool,break,case,
    class,compl,const,constexpr,continue,decltype,default,delete,
    do,double,else,enum,explicit,false,for,friend,goto,if,inline,
    int,long,mutable namespace,new noexcept,operator,private,
    protected,public,register,return,short,signed,sizeof,
    static,static_assert,static_cast,struct,switch,template,this,
    thread_local,throw,true,try,typedef,typeid,typename,union,
    unsigned,using,virtual,void,volatile,wchar_t,while
  },
  sensitive=true,
  morecomment=[l]//,
  morecomment=[s]/* */ ,
  morestring=[b]",
}

% Java
\lstdefinelanguage{Java}{
  morekeywords={
    abstract,assert,boolean,break,byte,case,catch,char,class,
    const,continue,default,do,double,else,enum,extends,final,
    finally,float, for,goto,if,implements,import,instanceof,int,
    interface,long,native,new, null,package,private,protected,
    public,return,short,static,strictfp, super,switch,
    synchronized,this,throw,throws,transient,true,try,void,
    volatile,while
  },
  sensitive=true,
  morestring=[b]",
}

% Go
\lstdefinelanguage{Go}{
  morekeywords={
    break,case,chan,const,continue,
    default,defer,else,fallthrough,
    for,function,goto,if,import,interface,
    map,package,range,return,select,struct,
    switch,type,var},
  sensitive=true,
  morecomment=[l]//,
  morecomment=[s]/* */ ,
  morestring=[b]",
}

% PHP
\lstdefinelanguage{PHP}{
  morekeywords={
    __halt_compiler,abstract,alias,arguments,break,case,class,
    clone,const,continue,declare,default,die,do,echo,else,elseif,
    empty,endswitch,eval,exit,extends,final,finally,for,foreach,
    function,global,goto,if,implements,include,include_once,
    instanceof,insteadof,interface,is,isset,list,namespace,
    print,private,protected,public,return,static,switch,throw,
    trait,try,unset,use,var,while,yield
  },
  sensitive=true,
  morecomment=[l]//,
  morecomment=[s]/* */ ,
  morestring=[b]",
}

% Javascript
\lstdefinelanguage{JavaScript}{
    morekeywords={abstract,arguments,await,boolean,break,byte,
    case,catch,class,const,continue,debugger,default,delete,do,
    double,else,enum,eval,export,extends,false,finally,for,function,
    global,if,implements,import,in,instanceof,int,let,match,namespace,
    NaN,private,protected,public,return,super,switch,throw,throws,true,
    try,typeof,var,void,yield
  },
  sensitive=true,
  morecomment=[l]//,
  morecomment=[s]/* */ ,
  morestring=[b]",
}



\geometry{margin=1cm}
\graphicspath { {./img/} }

\pagenumbering{gobble}
\date{}

\begin{document}

Radinal Shidiq Saragih

IF C 2023

5520123104

\begin{center}
  \section*{RESUME TUTORIAL CSS W3SCHOOL}
\end{center}

\section*{Karakteristik}

\begin{itemize}
  \item CSS adalah singkatan dari Cascading Style Sheets.
  \item CSS mendefinisikan bagaimana elemen html ditampilkan
  \item CSS dapat digunakan pada lebih dari satu halaman web
  \item CSS dapat disimpan secara eksternal dengan file .css.
\end{itemize}

\subsection*{Cara Menambahkan File Stylesheet CSS}

\begin{lstlisting}
<!DOCTYPE html>
<html>
  <head>
    <link rel="stylesheet" href="mystyle.css">
  </head>
  <body>

    <h1>This is a heading</h1>
    <p>This is a paragraph.</p>

  </body>
</html>
\end{lstlisting}

\section*{Syntax}

\begin{lstlisting}
  h1 { color: blue; font-size: 12px }
  /* selector(s) { declaration; }*/
\end{lstlisting}

\subsection*{Selector}

\begin{itemize}
  \item Selector Simple

    Bedasarkan id(#), nama, class(.).

  \item Selector Combinator

    Bedasarkan relasi, kombinator turunan (spasi), kombinator anak ($>$),
    kombinator next-sibling (+), kombinator subsquent-sibling(-)

  \item Selector Pseudo-class

    Bedasarkan sebuah keadaan/state.

    \begin{lstlisting}
    selector:pseudo-class {
      property: value;
    }
    \end{lstlisting}

  \item Selector Pseudo-element

    Bedasarkan area spesifik elemen.

    \begin{lstlisting}
    selector::pseudo-element {
      property: value;
    }
    \end{lstlisting}

  \item Selector Attribut

    Bedasarkan atribut khusus.

    \begin{lstlisting}
    a[target] {
      background-color: yellow;
    }
    \end{lstlisting}

\end{itemize}

\subsection*{Warna}

\begin{itemize}
  \item RGB(A)

    \begin{lstlisting}
      rgb(255,255,255);
      rgba(255,255,255,255);
    \end{lstlisting}

  \item HEX

    \begin{lstlisting}
      #FFFFFF
    \end{lstlisting}

  \item HSL(A)
    \begin{lstlisting}
      hsl(9,100%,64%)
      hsla(9,100%,64%,0.5)
    \end{lstlisting}
\end{itemize}

\subsection*{Background}
Styling untuk latar belakang halaman

warna latar belakang.

\begin{lstlisting}
body {
  background-color: #000000;
}
\end{lstlisting}

gambar untuk latar belakang (berulang).

\begin{lstlisting}
body {
  background-image: url("background.png");
}
\end{lstlisting}

untuk mengatur arah gambar diulang gunakan background-repeat.

\begin{lstlisting}
body {
  background-image: url("background.png");
  background-repeat: repeat-x; /* horizontal */
  background-repeat: repeat-y; /* vertical */
  background-repeat: no-repeat; /* jangan berulang */
}

\end{lstlisting}

untuk mengatur lokasi gambar gunakan background-position

\begin{lstlisting}
body {
  background-image: url("background.png");
  background-repeat: repeat-x;
  background-position: center;
  /*
    left top
    left center
    left bottom
    right top
    right center
    right bottom
    center top
    center center
    center bottom
    x% y% (lokasi relative)
    xpos ypos (pixel)
    init (default)
    (inherit) dari parent
  */
}

\end{lstlisting}

untuk mengatur scrolling

\begin{lstlisting}
body {
  background-image: url("bg.png");
  background-repeat: no-repeat;
  background-attachment: fixed;
  /*
    scroll -> scroll dengan halaman.
    fixed -> tidak akan scroll.
    local -> scroll dengan konten elemen.
  */
}

\end{lstlisting}

format background singkat

\begin{lstlisting}
body {
    background: #ffffff url("img_tree.png") no-repeat right top;
    /* background-color background-image background-repeat background-attachment background-position*/
}

\end{lstlisting}

\subsection*{Border}

Jenis Border
\begin{lstlisting}
border-style: dotted;
border-top-style: dotted;
border-right-style: solid;
border-bottom-style: dotted;
border-left-style: solid;
/*
    dotted
    dashed
    solid 
    double
    groove
    ridge 
    inset 
    outset
    none
    hidden
*/
\end{lstlisting}

Ukuran Border
\begin{lstlisting}
border-width: thick;
/*
  thick, medium, thin
  (px, pt, cm, em)
*/
\end{lstlisting}

Warna Border
\begin{lstlisting}
border-color: red; 
border-color: red green blue yellow;
          /*ATAS, KANAN, BAWAH, KIRI*/
\end{lstlisting}

Border Rounded
\begin{lstlisting}
border-radius: 5px;
\end{lstlisting}

Format Singkat
\begin{lstlisting}
border: 5px solid red;
    /* width style color*/
\end{lstlisting}

\subsection*{Margin}

\begin{lstlisting}
margin-top: 100px;
margin-bottom: 100px;
margin-right: 150px;
margin-left: 80px;

margin: 25px 50px 75px 100px;
/* atas kanan bawah kiri */

margin: 25px 50px 100px;
/* atas kanan-kiri bawah  */

margin: 25px 100px;
/* atas-bawah kanan-kiri */

margin: 25px;
/* atas-bawah-kanan-kiri */

margin: auto;
/* tengahkan elemen secara horizontal */
\end{lstlisting}

\subsection*{Padding}

\begin{lstlisting}
padding-top: 100px;
padding-bottom: 100px;
padding-right: 150px;
padding-left: 80px;

padding: 25px 50px 75px 100px;
/* atas kanan bawah kiri */

padding: 25px 50px 100px;
/* atas kanan-kiri bawah  */

padding: 25px 100px;
/* atas-bawah kanan-kiri */

padding: 25px;
/* atas-bawah-kanan-kiri */

padding: auto;
/* tengahkan elemen secara horizontal */
\end{lstlisting}

\subsection*{Height dan Width}

\begin{lstlisting}
div {
  height: 200px;
  width: 50%; /* % -> bedasarkan blok terluar */
  max-width: 500px; /* limit ukuran elemen */
  background-color: powderblue;
}
\end{lstlisting}

\subsection*{Outline}

Jenis Outline
\begin{lstlisting}
outline-style: dotted;
/*
    dotted
    dashed
    solid 
    double
    groove
    ridge 
    inset 
    outset
    none
    hidden
*/
\end{lstlisting}

Ukuran Outline
\begin{lstlisting}
outline-width: thick;
/*
  thick, medium, thin
  (px, pt, cm, em)
*/
\end{lstlisting}

Warna Outline
\begin{lstlisting}
outline-color: red; 
outline-color: red green blue yellow;
          /*ATAS, KANAN, BAWAH, KIRI*/
\end{lstlisting}


Offset Outline
\begin{lstlisting}
outline-offset: 15px;
/*
  OFFSET DILUAR BORDER
  (px, pt, cm, em)
*/
\end{lstlisting}

\subsection*{Teks}

\begin{lstlisting}
/* alignment teks */
text-align: center; 
text-align-last: left;
/* center, left, right, justify */

/* arah teks */
direction: rtl; /* ltr rtl */
unicode-bidi: bidi-override;

vertical-align: baseline; 
/*
  baseline
  text-top
  text-bottom
  sub
  super
*/

text-decoration-line: overline; /* overline line-trough underline */
text-decoration-color: red;
text-decoration-style: solid;
/*
    dotted
    dashed
    solid 
    double
    groove
    ridge 
    inset 
    outset
    none
    hidden
*/
text-decoration-thickness: 5px;

text-decoration: underline red double 5px;
/* line color style thickness */

text-transform: lowercase; /* uppercase lowercase capitalize */

text-indent: 5px; /* indentasi baris pertama */
letter-spacing: 5px; /* jarak antara karakter */
word-spacing: 5px; /* jarak antara kata */
line-hegith: 5px; /* jarak antara baris */
white-space: nowrap; /* pengaturan whitespace */

text-shadow: 2px 2px 5px red; /* horizontal-shadow vertical-shadow blur warna */

font-family: Arial, Helvetica;
/* font web-safe
  Arial (sans-serif)
  Verdana (sans-serif)
  Tahoma (sans-serif)
  Trebuchet MS (sans-serif)
  Times New Roman (serif)
  Georgia (serif)
  Garamond (serif)
  Courier New (monospace)
  Brush Script MT (cursive)
*/

font-style: normal; /* normal italic oblique */
font-weight: bold; /* normal bold */
font-variant: normal /* normal small-caps */
font-size: 5px; /* % em px */
\end{lstlisting}

\subsection*{Link}
\begin{lstlisting}
a:link    /* belum dikunjungi */
a:visited /* sudah dikunjungi */
a:hover   /* ketika cursor diatas link */
a:active  /* ketika diclick */
\end{lstlisting}

\subsection*{List}
\begin{lstlisting}
/* bentuk marker list */
list-style-type: circle; /* circle square upper-roman lower-alpha none */

/* gambar sebagai marker */
list-style-image: url('icon.png')

/* posisi marker */
list-style-position: outside; /* outside inside */
\end{lstlisting}

\subsection*{Tabel}
\begin{lstlisting}
border-collapse: collapse; /* gabungkan border menjadi satu border */
\end{lstlisting}

\subsection*{Position}
\begin{lstlisting}
position: static; /* static relative fixed absolute sticky */
\end{lstlisting}

\subsection*{Gambar}
\begin{lstlisting}
z-index: -1; /* ada dibelakang */
z-index: 1; /* ada didepan */
\end{lstlisting}

\section*{Box Model}
\begin{itemize}
  \item Content - Area Konten.
  \item Padding - Area pembatas sekeliling konten.
  \item Border - Area pembatas sekeliling konten dan padding.
  \item Margin - Area pembatas disekeliling border.
\end{itemize}

\end{document}
