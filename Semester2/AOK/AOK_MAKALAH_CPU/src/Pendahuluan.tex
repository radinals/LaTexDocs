\subsection{Latar Belakang}

Dalam era perkembangan teknologi informasi yang pesat, pemahaman terhadap
Central Processing Unit (CPU) menjadi sangat krusial. CPU, atau sering
disebut prosesor, merupakan otak dari setiap sistem komputer. Sebagai komponen inti,
CPU bertanggung jawab untuk menjalankan instruksi-instruksi yang diberikan
oleh perangkat lunak dan mengendalikan operasi-operasi dasar pada tingkat tinggi.

Kemampuan dan kinerja CPU memiliki dampak langsung terhadap kecepatan dan
efisiensi suatu sistem komputer. Oleh karena itu, pemahaman yang mendalam
terhadap arsitektur dan organisasi CPU sangat penting, terutama bagi mahasiswa
yang mengambil mata kuliah Arsitektur dan Organisasi Komputer.

Dengan adanya pemahaman yang mendalam terhadap CPU, diharapkan pembaca
dapat mengaplikasikan pengetahuan ini dalam konteks yang lebih luas,
seperti pengembangan perangkat lunak, perancangan sistem, dan pemecahan
masalah dalam dunia teknologi informasi. Melalui eksplorasi terhadap CPU,
diharapkan pula bahwa mahasiswa dapat membangun dasar yang kuat untuk pemahaman
lebih lanjut dalam bidang komputer dan teknologi informasi.

\subsection{Tujuan Masalah}

\begin{enumerate}[label=\alph*.]
  \item Terdiri dari apa sajakah suatu CPU?
  \item bagaimana bagian-bagian suatu CPU berinteraksi dan bekerja sama?
  \item Apa yang dimaksud dengan operasi mikro?
  \item Apa saja jenis operasi mikro?
\end{enumerate}
