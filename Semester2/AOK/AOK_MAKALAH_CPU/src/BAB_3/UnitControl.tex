Unit Control CPU adalah bagian dari CPU yang sangat penting dalam memastikan CPU
berjalan sebagaimana seharusnya.

Dalam suatu organisasi suatu CPU, unit control tersebut mengatur tugas antara
bagian CPU, agar instruksi yang didapatkan CPU dapat dijalankan secara sistematis.

Unit ini terdiri atas beberapa sub-unit atau elemen yang masing-masing memiliki
fungsi dan tugas tersendiri.

Menurut Prasetyo B., Puspitasari, A., dan Nasution, R, (2019), Unit Control adalah ``Elemen Control unit merujuk pada bagian dari sebuah sistem komputer yang bertanggung jawab
untuk mengarahkan dan mengkoordinasikan aktivitas seluruh unit fungsional dalam pemrosesan data``.

Berikut adalah unit internal yang umumnya terdapat dalam Unit Control CPU.

\begin{enumerate}[label=\alph*.]

  \item \textbf{ Control Lines}

    Sinyal-sinyal yang mengatur operasi dalam suatu unit pemrosesana, contohnya
    proses \textit{read/write} atau untuk mengirim sinyal agar mengaktifkan ALU
    (Arithmetic Logic Unit).

  \item \textbf{ ALU Control Unit}

    Bagian yang mengontrol operasi-operasi aritmatika yang dilakukan ALU
    (penambahan, pengurangan, perkalian, atau operasi logika).


  \item \textbf{ Clock Circuit }

    Bagian yang mengatur alur waktu dalam pengeksekusian instruksi, agar
    dieksekusi dengan urutan yang tepat.

  \item \textbf{ Bus Interface Unit }

    Bagian yang berada diantara CPU dengan sistem bus komputer, dan bagian inilah
    yang memberikan CPU akses ke komputer secara keseluruhan, seperti RAM, ataupun
    perangkat I/O.

  \item \textbf{ Control Unit Sequencer }

    Bagian yang memastikan langkah-langkah instruksi memiliki urutan yang benar
    agar dapat dieksekusi dengan tepat (Sequencing Logic).

  \item \textbf{ Decoder }

    Decoder atau Unit Decoder, adalah bagian Control Unit yang berfungsi
    untuk memproses instuksi tersebut dan menghasilkan sinyal-sinyal kontrol
    yang sesuai.

\end{enumerate}
