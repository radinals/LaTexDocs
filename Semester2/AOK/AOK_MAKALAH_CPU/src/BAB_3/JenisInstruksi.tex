Karena tugas dari prosesor adalah untuk menjadi otak dari segala operasi didalam komputer,
terdapat berbagai jenis instruksi yang diterimanya. Berikut adalah beberapa jenisnya.
\begin{itemize}

  \item Instruksi Aritmatika

    Menyajikan instruksi-instruksi yang digunakan untuk operasi aritmatika,
    seperti penambahan, pengurangan, perkalian, dan pembagian.

  \item Instruksi Logika:

    Menggambarkan instruksi-instruksi yang melakukan operasi logika pada data,
    seperti operasi bitwise AND, OR, XOR, dan NOT.

  \item Instruksi Transfer Data:

    Menjelaskan instruksi-instruksi yang digunakan untuk mentransfer data
    antara register, memori, atau perangkat lain.

  \item Instruksi Kontrol Alur Eksekusi:

    instruksi-instruksi yang mengendalikan alur eksekusi program, termasuk
    instruksi percabangan (branching), lompatan (jump), dan pemanggilan fungsi.

\end{itemize}
