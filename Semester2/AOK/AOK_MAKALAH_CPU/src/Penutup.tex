\subsection{Kesimpulan}

Kesimpulan yang didapat adalah, dalam CPU atau Central Processing Unit terdapat
tiga unit terpenting yaitu,

\begin{enumerate}[label=\alph*.]
  \item Control unit
  \item Arithmetic/Logic Unit (ALU)
  \item Memory Unit
\end{enumerate}

Dengan control unit bertugas untuk mengatur dua unit lainnya, Arithmetic/Logic Unit
atau ALU bertugas untuk melakukan perhitungan-perhitungan yang diperlukan dan proses
ini dibantu dengan Memory Unit yang terdiri dari cache dan register, fungsi utama dari
unit ini adalah untuk menyediakan tempat penyimpanan untuk data-data yang sedang di proses CPU.

Dan di tingkat mikro istilah operasi mikro yang mengacu kepada operasi tingkat
bit yang terjadi di Control Unit yang dihasilkan oleh proses decoding terhadap opcode atau
Operation-Code yang diterima Control Unit dari perangkat lunak komputer.

\subsection{Saran}

Agar lebih memahami operasi yang berada di tingkat CPU, perlu juga dipelajari
pemograman di tingkat rendah yang berbentuk assembly. Dengan memahami proses
eksekusi program di tingkat assembly, pemahaman bagaimana sebuah instruksi diproses
menjadi opcode akan didapatkan.
