Terdapat beberapa jenis operasi mikro, antara lain.

\begin{enumerate}[label=\alph*]

  \item \textbf{Operasi Aritmatika}
    \begin{enumerate}[label=\roman*.]
      \item Penambahan (ADD): Melibatkan penjumlahan dua operand.
      \item Pengurangan (SUB): Melibatkan pengurangan satu operand dari operand lain.
      \item Perkalian (MUL): Melibatkan perkalian dua nilai atau register.
      \item Pembagian (DIV): Melibatkan pembagian satu nilai oleh yang lain.
    \end{enumerate}

  \item \textbf{Operasi Bitwise Logika}
    \begin{enumerate}[label=\roman*.]
      \item AND Logika (AND): Melibatkan operasi AND bit-wise antara dua operand.
      \item OR Logika (OR): Melibatkan operasi OR bit-wise antara dua operand.
      \item XOR Logika (XOR): Melibatkan operasi XOR bit-wise antara dua operand.
      \item NOT Logika (NOT): Melibatkan operasi NOT bit-wise pada suatu operand.
    \end{enumerate}

  \item \textbf{Operasi Pergeseran Bit (Bitshift)}
    \begin{enumerate}[label=\roman*.]
      \item Pergeseran Kiri (Shift Left): Menggeser bit-bit suatu operand ke kiri.
      \item Pergeseran Kanan (Shift Right): Menggeser bit-bit suatu operand ke kanan.
    \end{enumerate}

  \item \textbf{Operasi Perbandingan}
    \begin{enumerate}[label=\roman*.]
      \item Operasi Setara (Equal): Membandingkan apakah dua operand setara.
      \item Operasi Kurang dari (Less Than): Membandingkan apakah satu operand kurang dari operand lain.
    \end{enumerate}

  \item \textbf{Operasi Load/Store}
    \begin{enumerate}[label=\roman*.]
      \item Load (Memuat): Memuat nilai dari suatu lokasi memori ke dalam register.
      \item Store (Menyimpan): Menyimpan nilai dari register ke dalam lokasi memori.
    \end{enumerate}

  \item \textbf{Operasi Kontrol Aliran}
  \begin{enumerate}[label=\roman*.]
    \item Operasi Skala (Jump): Melibatkan lompatan ke alamat tertentu berdasarkan kondisi tertentu.
    \item Operasi Panggil (Call): Melibatkan pemanggilan suatu subrutin atau fungsi.
  \end{enumerate}

  \item \textbf{Operasi Pemindahan Data}
  \begin{enumerate}[label=\roman*.]
    \item Pindah Register ke Register: Memindahkan nilai dari satu register ke register lain.
    \item Pindah Memori ke Register: Memuat nilai dari lokasi memori ke dalam register.
  \end{enumerate}

\item \textbf{Operasi Kontrol Aliran}
    \begin{enumerate}[label=\roman*.]
      \item Membaca Memori (Memory Read): Membaca nilai dari lokasi memori.
      \item Menulis ke Memori (Memory Write): Menulis nilai ke dalam lokasi memori.
    \end{enumerate}

  \item \textbf{Operasi Pemrosesan Floating-Point}
    \begin{enumerate}[label=\roman*.]
      \item  Operasi Aritmatika Floating-Point: Melibatkan operasi aritmatika untuk nilai floating-point.
    \end{enumerate}

  \item \textbf{Operasi Pemrosesan String}
  \begin{enumerate}[label=\roman*.]
    \item Manipulasi String: Melibatkan operasi penggabungan dan pemisahan karakter dalam string.
  \end{enumerate}

  \item \textbf{Operasi Kontrol Sirkuit Internal}
  \begin{enumerate}[label=\roman*.]
    \item  Pengaturan dan Kontrol Sirkuit Internal CPU: Mengendalikan berbagai elemen dan sirkuit internal di dalam CPU.
  \end{enumerate}

\end{enumerate}
